% corrected LN 74

\subsection{Design}

% Keras api definition

\textit{Keras} is a high-level Neural Networks API written in Python. It is
designed for easy-to-use and efficient experimentation with Deep
Learning.~\cite{chollet2015keras}\\

% getting started

The main data structure in Keras is a model. The common model is the
\textit{Sequential} model which organizes layers in a linear stack. An example
of a \textit{Sequential} model in Python is given below:

\begin{lstlisting}
from keras.models import Sequential

model = Sequential()
\end{lstlisting}

\noindent To stack layers in the model, the \textit{.add()} function is used:

\begin{lstlisting}
from keras.layers import Dense

model.add(Dense(unit=64, activation='relu', 
                input_dim=100))
model.add(Dense(unit=10, activation='softmax'))
\end{lstlisting}

\noindent After stacking up the model, its learning process has to be configured
with \textit{.compile()} function:

\begin{lstlisting}
modl.compile(loss='categorical_crossentropy', 
             optimizer='sgd',
             metrics=['accuracy'])
\end{lstlisting}

\noindent Now the training data can be iterated in batches:

\begin{lstlisting}
model.fit(x_train, y_train, epochs=5, batch_size=32)
\end{lstlisting}

\noindent To evaluate the performance of the model, the \textit{.evaluate()}
function is called:

\begin{lstlisting}
loss_and_metrics = model.evaluate(x_test, y_test, batch_size=128)
\end{lstlisting}
