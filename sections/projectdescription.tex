\section{Project description}
\subsection{Domains}

\begin{itemize}
        \item Speech Recognition
        \item Artifictial Neural Networks
        \item Deep Learning 
        \item Data preprocessing
        \item Training set and testing set
        \item Python
        \item Keras
\end{itemize}

\subsubsection{Scientific}

The scientific aspects covered by this Bachelor Semester Project are the
concepts of speech recognition and Deep Learning. Different Neural Networks
architectures are presented scientifically. \\

\textbf{Speech Recognition.} The objective of speech recognition is to map an
audio signal which contains a set of spoken natural language expressions to the
according sequence of words produced by the speaker.\\

\textbf{Artifictial Neural Networks (ANN).} ANNs are computing systems inspired
by the biological brain. These systems are based on a set of connected units
called artificial nodes. Each connection can transmit \textit{signals} between
units. A unit is able to process the signal and transmit it to another unit.\\

\textbf{Deep Learning.} This field deals with learning by decomposing the task's
input into smaller and simpler compositions. With Deep Learning, computing
systems can build complex concepts from a composition of simpler concepts.

\subsubsection{Technical} The technological aspect which is covered in this
project is the data collection, feature extraction and implementation of our
classification model. \\

\textbf{Data preprocessing.} Data preprocessing is an important phase in machine
learning. It ensures the quality of the gathered data by eliminating irrelevant
and redundant information. Datapreprocessing contains the tasks such as
cleaning, instance selection, normalization, feature extraction and seclection.
The result of data preprocessing is the training set. We will focus on feature
extraction in this paper with the presentation of Mel-frequency cepstral
coefficients in section.\\

%add section

\textbf{Training set and testing set.} Brief explanation about the data set.\\

\textbf{Python.} This is a programming language which is interpreted, high-level
and general-purpose.~\cite{Python}\\

\textbf{Keras Library.} Brief explanation about this python library.\\
