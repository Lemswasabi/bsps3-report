%corrected

\section{Project description}
\subsection{Domains}

\begin{itemize}
        \item Speech Recognition
        \item Artificial Neural Networks
        \item Deep Learning 
        \item Data preprocessing
        \item Training set and testing set
        \item Python
        \item Keras
\end{itemize}

\subsubsection{Scientific}

The scientific aspects covered by this Bachelor Semester Project are the
concepts of speech recognition and Deep Learning. Different Artificial Neural
Networks architectures are presented scientifically. \\

\textbf{Speech Recognition.} The objective of speech recognition is to map an
audio signal which contains a set of spoken natural language expressions to the
matching sequence of words produced by the speaker. In the past, Automatic
Speech Recognition (ASR) was made up of different modules such as complex
feature extraction, acoustic models, sequential models, language and
pronunciation models.~\cite{DBLP:journals/corr/AmodeiABCCCCCCD15} Sequential
models, such as Hidden Markov Models (HMM), in combination with a pre-trained
language model were used to map sequences of phones to output
words.~\cite{Williamsong} A different approach is to build ASR models
end-to-end. With deep learning, it replaces most of the modules with a single
module. This alternative method will be the main task of this report.\\

\textbf{Artifictial Neural Networks (ANN).} ANNs are computing systems inspired
by the biological brain. These systems are based on a set of connected units
called artificial nodes. Each connection can transmit \textit{signals} between
units. A unit can process the signal and transmit it to another unit.\\

\textbf{Deep Learning.} This field deals with learning by decomposing a task's
input into smaller and simpler compositions. With Deep Learning, computing
systems can build complex concepts from a composition of simpler concepts.

\subsubsection{Technical} The technological aspect which is covered in this
project is the data collection, feature extraction and implementation of our
classification model. \\

\textbf{Data preprocessing.} Data preprocessing is an important phase in machine
learning. It ensures the quality of the gathered data by eliminating irrelevant
and redundant information. Data preprocessing contains tasks such as cleaning,
instance selection, normalization, feature extraction and selection. The result
of data preprocessing is the training set. We will focus on feature extraction
in this paper with the presentation of Mel-frequency cepstral coefficients
(MFCCs) in section~\ref{mfccs}.\\

%add section

\textbf{Training set and testing set.} For a computing system to learn
from and make predictions on data, a mathematical model is built from an input
data. This input data used to create the model consists of two datasets. The
training and testing set. The training set contains pairs of an input vector and
an output vector often called the label. With the training set, the model learns
to map the input vector to the label. Whereas, the testing set evaluates how
well the model generalizes the prediction over the dataset.\\

\textbf{Python.} This is a programming language which is interpreted, high-level
and general-purpose.~\cite{Python}\\

\textbf{Keras Library.} \textit{Keras} is a high-level Neural Networks API
written in Python. It is designed for fast experimentation with Deep
Learning.~\cite{chollet2015keras}
