% corrected VD 93

\begin{abstract}

Luxembourgish is a West Germanic language which is mainly spoken in Luxembourg.
It is spoken by roughly 390000 people worldwide. Luxembourgish remains one of
Europe's under-described and under-resourced languages. In this paper, the focus
will be on isolated Luxembourgish word recognition. End-to-end automatic speech
recognition (ASR) is the current state-of-the-art. Deep learning replaces most
modules, such as language models or audio models, which were used in previous
ASR with a single model. To the best of our knowledge, there exist no end-to-end
continuous speech recognition for the Luxembourgish language. Thus efforts were
made on collecting a small audio dataset containing Luxembourgish spoken words
which are speaker-dependent. We apply RNN and feedforward Neural Networks as the
baseline for the recognition and compare it to more sophisticated RNN models.
Recurrent Neural Networks (RNN) use their internal state to operate on data
time-series. Combined with connectionist temporal classification (CTC) models,
they represent the current state-of-the-art in ASR. The accuracy to recognize
Luxembourgish words obtained with gated RNNs, such as long short-term memory
(LSTM) and gated recurrent units (GRU), were both roughly 0.99\%

\end{abstract}
