% corrected VD 81

\subsubsection{FR01: Feature extraction with MFCCs}\label{mfccs}~\\

% introduction

Before training end-to-end ASR model, audio data have to be preprocessed. The
first task is to extract features from the data which describes natural language
expressions and excludes background noises and emotions.\\

% def of MFCCs

Mel Frequency Cepstral Coefficients (MFCCs) are features widely used in ASR.
MFCCs were presented by Davis and Mermelstein in the 1980s.~\cite{mfcc}\\

To extract the MFCCS from the dataset, the following extraction steps have to be
executed on a speech signal sampled at $16kHz$:\\

\begin{enumerate}[label=\arabic*.]
  \item Frame the signal into $25ms$ frames. The frame length of a $16kHz$
    signal:
  \begin{equation} 
    0.025*16000 = 400 samples.
  \end{equation}
  With a frame step of $160\ samples$ or $10ms$, the frames overlap themselves,
  such that the first frame containing $400\ samples$ starts at sample 0 and the
  second frame starts at sample 160. This pattern repeats until the end of the
  speech signal. If the audio file is not divisible by an even number, it is
  padded by zeros.\\
  \item For each frame calculate the periodogram estimate of the power spectrum.
    To obtain the Discrete Fourier Transform (DFT) from the frame $i$, we
    perform:
    \begin{equation}
      S_{i}(k) = \sum_{N}^{n=1}s_i(n)h(n)e^{-j2\pi kn/N}
    \end{equation}
    For each speech frame $s_{i}(n)$, the periodogram-based power spectral
    estimate is calculated with:
    \begin{equation}
      P_{i}(k)=\frac{1}{N}{\mid{S_{i}(k)}\mid}^{2}
    \end{equation}
  This is called the Periodogram estimate of the power spectrum which identifies
  for every frame which frequencies are present.\\
  \item Apply the Mel filterbank to the power spectrum and sum the energy in
    each filter. The Mel filterbank is a set of 26 triangular filters. The
    filterbank energies are calculated by multiplying every filterbank with the
    power spectrum.\\
  \item Take the logarithm of the 26 filterbank energies.\\
  \item Take the Discrete Cosine Transform (DCT) of the 26 log filterbank
    energies resulting in 26 cepstral coefficients. We only used the lower 12-13
    of the 26 coefficients for ASR.\\
\end{enumerate}

In Section~\ref{librosa} we will use the Python library \textit{Librosa} to
extract the MFCCs features from the dataset.
