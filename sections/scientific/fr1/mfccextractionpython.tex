% corrected VD 78

\subsubsection{Feature extraction with MFCC in Python}~\label{librosa}~\\

For this experiment, we use an audio dataset of English spoken
words\cite{DBLP:journals/corr/abs-1804-03209}. From this dataset we choose the
words $w \in \{'zero', 'one', 'two', \dots , 'eight', 'nine'\}$. This dataset
contains many persons pronouncing these different words.  Every file is roughly
$1s$ long and is sampled at $16kHz$.\\

We use the Python library \textit{librosa}~\cite{librosa} to extract the MFCCs from the
dataset. Instead of processing the whole dataset, the following steps are
applied to one single audio file $f$.\\

\begin{enumerate}[label=\arabic*.]
  \item Let $f$ be an audio file sampled at $16kHz$, \textit{Librosa} loads the
song as follows:
\begin{lstlisting}
import librosa

y, sr = librosa.load(f, sr=16000, mono=True, duration=1)
\end{lstlisting}
with $y$ being the audio time series and $sr$ the sample rate of $y$.\\
\item The next step is to pad the time series $y$ if it is smaller than $1s$:
\begin{lstlisting}
import numpy as np

if y.size < 16000:
  rest = 16000 - y.size
  left = rest // 2
  right = rest - left
  y = np.pad(y, (left, right), 'reflect'))
\end{lstlisting}
The time series is reflected on both sides to get the correct size.\\
\item After matching the time series to the correct size, the MFCCs can be extracted
with:
\begin{lstlisting}
mfccs = librosa.feature.mfcc(y=y, sr=sr)
\end{lstlisting}
This function \textit{.mfcc\(\)} returns an \textit{np.ndarray} with the dimension:
\begin{equation*}
  shape=(n\_mfcc,t) 
\end{equation*}
with $n\_mfcc$ being the number of MFCCs and $t$ the number of frames.
\item The last step is to flatten this matrix to a vector which can be stored in a
  data frame for later training.
\begin{lstlisting}
row = mfcc.T.flatten() 
\end{lstlisting}
The matrix $mfcc$ is transposed and flattened. This concatenates the columns
together forming a vector.
\end{enumerate}~\\
Theses steps are then applied to the entire dataset. Each computed vector is
appended to the data frame which is going to be stored as a $.csv$ file. The
entire code snippet is in the Appendix in Figure~\ref{mfccsnip}.
