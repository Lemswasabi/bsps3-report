% 

\subsection{Assessment}
\subsubsection{NFR01: Performance evaluation}~\\

In this section, the performance of the four ANNs investigated in
Section~\ref{ann} are evaluated. We use the English and Luxembourgish dataset
for the performance analysis.\\

The first step is to choose an optimal optimizer. The choice is between using
the stochastic gradient descent (SGD) or the Adam optimizer. The second is a
combination of two extensions of SGD, Adaptive Gradient Algorithm (AdaGrad) and
Root Mean Square Propagation (RMSProp)~\cite{Adam}. The four models will be
trained on the english dataset to designate an optimal optimizer and number of
epochs. Their training will be visualized with $60\ epochs$ each. The training
vizualization are in the appendix as follows:\\

\begin{enumerate}[label=\arabic*.]
  \item Feedforward
    \begin{itemize}
      \item accuracy in Figure.\ref{ffa}
      \item loss in Figure.\ref{ffl}
    \end{itemize}
  \item RNN
    \begin{itemize}
      \item accuracy in Figure.\ref{rnna}
      \item loss in Figure.\ref{rnnl}
    \end{itemize}
  \item LSTM
    \begin{itemize}
      \item accuracy in Figure.\ref{lstma}
      \item loss in Figure.\ref{lstml}
    \end{itemize}
  \item GRU
    \begin{itemize}
      \item accuracy in Figure.\ref{grua}
      \item loss in Figure.\ref{grul}
    \end{itemize}
\end{enumerate}~\\
Let $n=30$ for the number of epochs. The two optimizers return the following
results:

\begin{table}[H]
    \centering
    \resizebox{\columnwidth}{!}{%
    \begin{tabular}{|c|c|c|c|c|c|c|c|c|}
        \hline
        \multirow{2}{*}{Optimizer} & \multicolumn{2}{c|}{FF} &
        \multicolumn{2}{c|}{RNN} & \multicolumn{2}{c|}{LSTM} &
        \multicolumn{2}{c|}{GRU} \\
        \cline{2-9}
        & {acc} & {loss} & {acc} & {loss} & {acc} & {loss} & {acc} & {loss}\\
        \hline
        SGD & 0.10 & NaN & 0.27 & 1.94 & 0.86 & 0.46 & 0.85 & 0.46 \\ \hline
        Adam & 0.77 & 0.24 & 0.30 & 0.24 & 0.94 & 0.24 & 0.94 & 0.24 \\ \hline
    \end{tabular}%
    \caption{Comparison of the four models' testing results using SGD and Adam.
    The models were trained on the English dataset.}
    \label{table:comparison}
    }
\end{table}

With SGD, the feedforward model didn't train and its loss gives $NaN$. However,
we still can choose an optimizer since the results using the Adam optimizer
yield better accuracies compared to SGD. For the following investigation, the
Adam optimizer is considered.\\

The next step is to choose the number of epochs which will be used for the
performance analysis. The graphs show the loss converges around $40$ epochs.
Therefore, we choose $epochs = 40$.\\

Let $epochs = 40$ we evaluate each model's performance, we have:

\begin{table}[H]
    \centering
    \resizebox{\columnwidth}{!}{%
    \begin{tabular}{|c|c|c|c|c|c|c|c|c|}
        \hline
        \multirow{2}{*}{Optimizer} & \multicolumn{2}{c|}{FF} &
        \multicolumn{2}{c|}{RNN} & \multicolumn{2}{c|}{LSTM} &
        \multicolumn{2}{c|}{GRU} \\
        \cline{2-9}
                                 & {acc} & {loss} & {acc} & {loss} & {acc} &
        {loss} & {acc} & {loss}\\ \hline
        Adam & 0.76 & 0.23 & 0.26 & 0.22 & 0.937 & 0.23 & 0.94 &
        0.23 \\ \hline
    \end{tabular}%
    \caption{The testing results of the four models using Adam. The four models
    were trained on the English dataset. $FF$ is the abbreviation for feedforward
    and $acc$ stands for accuracy.}
    \label{table:40}
    }
\end{table}

The results' inspection shows that LSTM and GRU perform better than feedforward
and RNN. The overall performance is slightly underneath the performance at 30
epochs. In this case, RNN has the lowest accuracy. This can be explained with
the vanishing and exploding gradients or because the learning rate was not set
optimally during training. As expected, LSTM and GRU perform better than LSTM
since they deal with the challenge of long-term dependencies. Compared to
feedforward, LSTM and GRU have better accuracy since they take advantage of
their recurrent units to process series data.\\

The following testing results were trained on the Luxembourgish dataset:

\begin{table}[H]
    \centering
    \resizebox{\columnwidth}{!}{%
    \begin{tabular}{|c|c|c|c|c|c|c|c|c|}
        \hline
        \multirow{2}{*}{Optimizer} & \multicolumn{2}{c|}{FF} &
        \multicolumn{2}{c|}{RNN} & \multicolumn{2}{c|}{LSTM} &
        \multicolumn{2}{c|}{GRU} \\
        \cline{2-9}
        & {acc} & {loss} & {acc} & {loss} & {acc} & {loss} & {acc} & {loss}\\
        \hline
        Adam & 0.35 & 0.02 & 0.56 & 0.02 & 1.0 & 0.02 & 0.99 & 0.02 \\ \hline
    \end{tabular}%
    \caption{The testing results of the four models using Adam. The four models
    were trained on the Luxembourgish dataset. The dataset contains 50 instances
    of a spoken word. FF is the abbreviation for feedforward and $acc$ stands for
    accuracy.}
    \label{table:lux}
    }
\end{table}

The results show that RNN, LSTM and GRU generalize better than feedforward.
LSTM and GRU are outperforming RNN in this results as well. LSTM is overfitting
during training. The high accuracy of LSTM and GRU can be explained by the small
dataset. The dataset contains only 50 instances of a spoken word and is speaker
dependent. The results of LSTM and GRU will drop if the dataset grows and
contains multiple speakers.

\textcolor{gray}{add results for bigger luxembourg dataset.}
