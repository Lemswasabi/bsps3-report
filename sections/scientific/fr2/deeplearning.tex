% corrected LN 87

\subsubsection{FR02: Deep Learning and Artificial Neural Networks}~\\

The notions of Deep Learning and examples of ANNs presented in this section are
based on the textbook \textit{Deep Learning}\cite{Goodfellow-et-al-2016} written
by \textit{Ian Goodfellow, Yoshua Bengio and Aaron Courville}. These
presentations should only give a high-level introduction to these notions.

\subsubsection{Deep Learning}~\\

Artificial Intelligence (AI) is a field with many practical applications such as
understanding speech, etc. AI deals with challenging problems which are easy to
perform but hard to describe formally by humans. Deep Learning is a solution to
these intuitive problems. This approach to AI allows computer systems to study
from experience and understand the world through a hierarchal stack of concepts.
The computing system gathers knowledge from experience which eliminates the need
to describe formal rules. The computer understands complex concepts with this
stack of concepts by decomposing them with simpler ones. The representation of
this hierarchy of concepts shows a deep graph with many layers, hence the name
for Deep Learning.~\cite{Goodfellow-et-al-2016}
